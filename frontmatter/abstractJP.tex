\chapter*{要旨}
%800字程度

本研究は、社会資本投資が地域経済に与える影響を実証的に分析することを目的とする。社会資本投資には、インフラ整備、公共サービス、地域コミュニティ支援などが含まれ、地域経済の成長と持続可能性における重要な要素とされている。本研究では、統計データと計量経済モデルを用いて、社会資本投資が地域のGDP成長、雇用率、所得分配に与える直接的および間接的な影響を明らかにする。分析の結果、社会資本投資は地域経済の活性化に寄与するだけでなく、地域間格差の是正にも一定の効果を持つことが示唆された。また、資源の戦略的配分が経済的利益を最大化するために不可欠であり、特に人口減少や財政制約が課題となる地方において、持続可能かつ包括的な成長を実現するための政策的示唆が得られた。これらの成果を踏まえ、社会資本投資の配分を最適化するための新たな指針や、地域経済政策への実践的な応用の可能性についても議論を行う。本研究の結果は、今後の社会資本投資戦略の策定や政策立案に貢献することが期待される。

