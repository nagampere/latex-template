\makeatletter % "@"記号の有効化

%%% セクション構成の設定 %%%
\renewcommand{\chapter}{% chapter番号の設定
\newpage
\@startsection{chapter}{1}{\z@}%
{0.0\Cvs \@plus.5\Cdp \@minus.2\Cdp}%
{1.8\Cvs \@plus.3\Cdp}%
{\reset@font\huge\sf}%
}
\renewcommand{\thesubsection}{ % subsecionの設定
  (\arabic{subsection})%
}

%%% 引用スタイルの設定 %%%
\renewcommand{\bibname}{参考文献} % 参考文献のタイトル
\setcitestyle{super,sort&compress,citesep={,},open={},close={)}}  % 参考文献の引用形式(natbib)
% \renewcommand{\bibsection}{\chapter*{\bibname}}

%%% その他の設定 %%%
\renewcommand{\baselinestretch}{1} % 行間の大きさ(倍率)
\setcounter{tocdepth}{2} % 目次のセクションの深さ(章(chapter), 節(section), 小節(subsection)の0~2段階)
\setcounter{MaxMatrixCols}{20} % 行列の最大列数
\pgfplotsset{width=\textwidth, height=0.6\textwidth, compat=1.18} % pgfplotsの図サイズ
\graphicspath{{figure/}}

\makeatother % "@"記号の有効化をオフ
