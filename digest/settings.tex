\makeatletter % "@"記号の有効化

\makeindex
\setcounter{tocdepth}{2}

%%% セクション構成の設定 %%%
\renewcommand{\section}{%
	\@startsection {section}{1}{\z@}% セクション表記の設定
	{-2.5ex plus -1ex minus -.2ex} % 前の行間を縮める
	{1.5ex plus .2ex} % 後の行間を縮める
	{\sf\bf}% セクションのフォント設定
}

\renewcommand{\subsection}{% サブセクション表記の設定
	\@startsection {subsection}{1}{\z@}%
	{-2.5ex plus -1ex minus -.2ex} % 前の行間を縮める
	{1.5ex plus .2ex}% 後の行間を縮める
	{\bf}% サブセクションのフォント設定
}


%%% ページレイアウトの設定 %%%
\setlength{\topmargin}{-7.4truemm} % 上の余白
\setlength{\oddsidemargin}{25mm} % 左の余白
\iftombow % トンボがある場合
	\addtolength{\oddsidemargin}{-1in} % トンボの分だけ左にずらす
\else
	\addtolength{\oddsidemargin}{-1truein} % トンボの分だけ左にずらす
\fi
\setlength{\headheight}{1mm} % ヘッダの高さ
\setlength{\headsep}{1mm} % ヘッダと本文の間隔
\setlength{\textwidth}{48zw} % 1行48文字(二段組の場合は24文字)
\setlength{\textheight}{46\baselineskip} % 1ページ42行程度(適宜微調整して下さい)
\addtolength{\textheight}{\topskip} % 本文とフッタの間隔
% \setlength{\parindent}{1em} % 段落のインデントを1文字分に設定


%%% ヘッダー・フッターの設定 %%%
\pagestyle{fancy}
\fancyhf{} % すべてのヘッダーとフッターをクリア
\renewcommand{\headrulewidth}{0.0pt} % ヘッダーの下の線を消す
\renewcommand{\footrulewidth}{0.0pt} % フッターの上の線を消す
% 1ページ目のスタイルを設定
\fancypagestyle{firstpage}{
	\fancyhf{} % すべてのヘッダーとフッターをクリア
	\renewcommand{\headrulewidth}{0.0pt} % ヘッダーの下の線を消す
	\renewcommand{\footrulewidth}{0.8pt} % フッターの上の線を太くする
	\fancyfoot[L]{{\small キーワード:社会資本投資,地域経済,計量経済モデル,GDP成長,所得分配}} % フッターの左側にキーワードを表示
}
% 2ページ目以降のスタイルを設定
\fancypagestyle{plain}{
	\fancyhf{} % すべてのヘッダーとフッターをクリア
	\renewcommand{\headrulewidth}{0.0pt} % ヘッダーの下の線を消す
	\renewcommand{\footrulewidth}{0.0pt} % フッターの上の線を消す
}


%%% 引用スタイルの設定 %%%
\renewcommand{\bibname}{参考文献} % 参考文献のタイトル
\setcitestyle{super,sort&compress,citesep={,},open={},close={)}}  % 参考文献の引用形式(natbib)
\setlength{\bibsep}{0pt} % 参考文献リストの行間を狭くする
% \renewcommand{\bibsection}{\chapter*{\bibname}}

%%% 参考文献リストの行間をさらに狭くする設定 %%%
\renewenvironment{thebibliography}[1]
{\section*{\refname\@mkboth{\refname}{\refname}}%
	\list{\@biblabel{\@arabic\c@enumiv}}%
	{\settowidth\labelwidth{\@biblabel{#1}}%
		\leftmargin0pt % インデントをゼロにする
		\advance\leftmargin\labelsep
		\setlength\itemsep{0pt} % 行間を狭くする
		\setlength\parsep{0pt} % 行間を狭くする
		\setlength\baselineskip{9pt} % 文字の大きさを調整
		\setlength{\labelsep}{0.5em} % ラベルと本文の間のスペースを調整
		\setlength{\labelwidth}{1em} % ラベルの幅を調整
		\@openbib@code
		\usecounter{enumiv}%
		\let\p@enumiv\@empty
		\renewcommand\theenumiv{\@arabic\c@enumiv}}%
	\sloppy
	\clubpenalty4000
	\@clubpenalty\clubpenalty
	\widowpenalty4000%
	\sfcode`\.\@m
	\setlength{\parindent}{0pt} % 参考文献リスト内のインデントを無効にする
}
{\def\@noitemerr
	{\@latex@warning{Empty `thebibliography' environment}}%
	\endlist}


%%% 数式の文字サイズを小さく設定 %%%
\everydisplay{\small} % ディスプレイ数式の文字サイズを小さくする

%%% 数式の文字間隔を狭く設定 %%%
\setlength{\thinmuskip}{0.5mu} % 通常の数式の間隔を狭くする
\setlength{\medmuskip}{1mu}  % 中程度の数式の間隔を狭くする
\setlength{\thickmuskip}{1mu} % 太い数式の間隔を狭くする


%%% その他の設定 %%%
\renewcommand{\baselinestretch}{1} % 行間の大きさ(倍率)
\setcounter{tocdepth}{2} % 目次のセクションの深さ(章(chapter), 節(section), 小節(subsection)の0~2段階)
\setcounter{MaxMatrixCols}{20} % 行列の最大列数
\pgfplotsset{compat=1.18}



\makeatother % "@"記号の有効化をオフ
